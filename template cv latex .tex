\documentclass[]{Vrais-CV-phalus}
%\documentclass[]{twentysecondcv}
%\usetikzlibrary{arrows}	

\begin{document}
	\profilepic{gu.jpg} %path of profile pic
	\cvname{Gurvan Huet} %your name
	\cvjobtitle{R\&D Mécanique analitique}%your actual job position
	
	\nationality{français}
	\birth{24 ans}
	\licenceone{Permis de conduire}
	\licencetwo{Permis bateau}
	\cvaddress{Paris}%address, 39 quater rue des vignes, 78480, Verneuil Sur Seine
	\cvnumberphone{06 52 48 45 17}%telphone number
	\cvmail{grvn.huet@gmail.com}%e-mail
	\linkdin{https://www.linkedin.com/in/gurvan-huet-422057b8/}

	\skilla{\textbf{Dvlps.} \\ \Large \textbf{projets}}
	\skillb{\textbf{Vif} \\ \Large \textbf{d'esprit}}
	\skillc{\textbf{Manager}}

	\logohobbya{logo-music.png}
	\legendehobbya{Batterie\&Violon\\Jam\&Improvisation}
	\logohobbyb{logo-pav.png}
	\legendehobbyb{10 ans de compétitions\\Participation à des régates internationales}
	\logohobbyc{logo-linux.png}
	\legendehobbyc{Port d'intérêts aux protocoles TCP-IP et \\aux logiciels open-sources}

	\languagea{france.png}
	\legendelanguagea{Langue maternelle}
	\languageb{united-kingdom.png}
	\legendelanguageb{Niveau C1\\Toeic(900)}
	\languagec{spain.png}
	\legendelanguagec{Niveau B1-B2\\7 mois au Chili}
	\languaged{germany.png}
	\legendelanguaged{Niveau A2-B1\\6 mois \\à Munich}

	\toola{sysml.png}
	\toolb{catia.png}
	\toolc{RRmatlab.png}
	\toold{latex.png}
	\toole{python.png}
	\toolf{office.png}

	\makeprofile
	\makeskills
	\maketools
	\makehobbies
	\makelanguages
	\maketit
	
	%\CreateTableau{3}{7}{Juillet}
	%tableau à longueur/largeur variable pour l'amélioration du cv (dire cb de skills ou languages directement dans interface users sans aller dans le cls)
	
	%Changer l'onglets skills, enlever la barre de com, la remplacer par un carré avec le num skill au dessus du rond

	\begin{minipage}{0.5\textwidth}
		
		%---------------------------------------------------------------------------------------
		%	EXPERIENCE / EDUCATION
		%----------------------------------------------------------------------------------------
		\centering\cvsect{Experience et Education}{0.4}{thirdcol}{titletext}\\[16pt]
		
		\vspace{-0.5cm}
		\begin{center}
			
			% TIMELINE
			\begin{cvtimeline}{2013}{2020}{16}
				
					\cv{7/2013}{9/2013/<-/,7/2014//dotted,9/2014//,7/2015//white,9/2015//,6/2018//white,7.3/2018//,12/2018//white,04/2019/->/}{c3}{0/-6.3}{bno.jpg}{2/3}{\textbf{BNO/Casa Bote} }{Moniteur de voile}{Pour enfants, adultes et handicapés}
					\cv{3/2015}{5/2015/<->/}{c3}{-0.15/-0.7}{alluresyachting.jpg}{2/3}{\textbf{Allures Yachting}}{Technicien}{Support à la production de bateaux de croisières}
					\cv{7/2017}{01/2018/<->/}{c2}{-0.15/0}{dlr.png}{2/2}{\textbf{\hspace{1mm}German Aerospace Center}}{Stage 6 mois: Chercheur}{Développement mécanique d'une prothèse de jambe à articulation souple. Implémentation de proriétés intrinsèques à la structure telles que l'absorbtion de l'énergie lors d'une chute et rendre la paire de jambe la plus stable possible. Mon travail a consisté à étudier et prototyper un modèle de prothèse capable de tenir debout et d'encaisser une force verticale, seul en position neutre.}

					\cv{9/2013}{7/2015/<->/}{c2}{0/-1.4}{Inewton.jpeg}{1/2}{\textbf{Lycée Newton}}{Prépa PT}{Etudiant en Physique Techniques}
					\cv{9/2015}{7/2017/<->/}{c2}{0/-3.4}{Arts-et-metiers.png}{1/2}{\textbf{Arts \& Métiers Lille}}{Tronc commun}{Ingénierie mécanique généraliste}
					\cv{12/2016}{6/2017/<->/}{c3}{0.15/-1.75}{edhec.jpg}{1/3}{\textbf{Edhec}, Cours \textbf{KAM}}{\textbf Key \textbf Account \ \ \textbf Management}{Management de comptes clients et fournisseurs. (50h)}
					\cv{1/2018}{6/2018/<->/}{c1}{0.15/0}{creda.jpeg}{1/1}{\textbf{Arts \& Métiers Paris}}{CREDA}{Spé. Entrepreneuriat, maitrise outils de dvlp. projets et de leurs applications dans le cas du projet "Energie des Moulins de France"(EMDF)}

					\cv{5/2013}{0}{bb}{0/-0.48}{}{6/1}{Baccalauréa scientifique,\\option mathématique}{}{}
					\cv{9/2018}{0}{bb}{0/2.23}{Arts-et-metiers.png}{5/1}{\textbf{Master} : \ \ \ \ \ \ \ \ \ \ \ \ \ Ingénierie Mécanique}{}{}

			\end{cvtimeline}
		\end{center}
	\end{minipage}
	\hspace{-2.5mm}\begin{minipage}{0.5\textwidth}
		
		%---------------------------------------------------------------------------------------
		%	Projets
		%----------------------------------------------------------------------------------------
		\centering\cvsect{Projets}{0.4}{thirdcol}{titletext}\\[16pt]
		
		\vspace{-0.5cm}
		\begin{center}
			
			% TIMELINE
			\begin{cvtimeline}{2013}{2020}{16}
				
				\cv{10/2015}{7/2016/<->/}{c1}{0/-2.9}{gu.jpg}{4/1}{\textbf{Guadz'Hum}}{Président de l'association. Organisation et réalisation d'un projet de restauration d'une école à Bhaktapur au Népal d'une durée d'un mois.}{}
				\cv{10/2016}{7/2017/<->/}{c3}{0/-1}{gu.jpg}{4/3}{\textbf{Voilz'Arts}}{Président de l'association. Organisation de cours de voiles en mer en bateaux habitables}{}
				\cv{10/2017}{8/2018/<->/}{c1}{0/0.4}{gu.jpg}{4/1}{\textbf{EMDF} (projet CREDA)}{Création d'un business plan et dvlp. technique d'une entreprise installant des centrales hydro-électriques dans des moulins à eau}{}
				
				\cv{8/2014}{9/2018/<-/,10/2019//dotted}{c2}{0/-5}{gu.jpg}{3/2}{\textbf{Trace rectangle}}{Dvlp. d'un outil permettant aux élèves de prépa de travailler de manière plus rapide lors des concours écrits à l'aide d'une règle multi-fonctions. \\ Brevet déposé}{}
				\cv{8/2017}{02/2018/<->/}{c1}{0.15/-1.5}{poeuf.png}{1/1}{\textbf{Poeuf}}{Projet tronc commun AM}{Dvlp. technique d'une couveuse biomimétique. Levée de fond en crowdfunding de 1500 euros. Prototype réalisé mais non compétitif vis à vis du prix de vente unité}
				\cv{9/2018}{5/2019/<->/}{c2}{0.15/0}{gu.jpg}{3/2}{\textbf{Chili}}{8 mois, apprentissage de l'espagnol et aide à la mise en route du club nautique Casa-Bote}{}
				
			\end{cvtimeline}
		\end{center}
	\end{minipage}



\end{document}
